\documentclass[journal,12pt,twocolumn]{IEEEtran}

\usepackage{setspace}
\usepackage{gensymb}
\singlespacing
\usepackage[cmex10]{amsmath}

\usepackage{amsthm}

\usepackage{mathrsfs}
\usepackage{txfonts}
\usepackage{stfloats}
\usepackage{bm}
\usepackage{cite}
\usepackage{cases}
\usepackage{subfig}

\usepackage{longtable}
\usepackage{multirow}

\usepackage{enumitem}
\usepackage{mathtools}
\usepackage{steinmetz}
\usepackage{tikz}
\usepackage{circuitikz}
\usepackage{verbatim}
\usepackage{tfrupee}
\usepackage[breaklinks=true]{hyperref}
\usepackage{graphicx}
\usepackage{tkz-euclide}

\usetikzlibrary{calc,math}
\usepackage{listings}
    \usepackage{color}                                            %%
    \usepackage{array}                                            %%
    \usepackage{longtable}                                        %%
    \usepackage{calc}                                             %%
    \usepackage{multirow}                                         %%
    \usepackage{hhline}                                           %%
    \usepackage{ifthen}                                           %%
    \usepackage{lscape}     
\usepackage{multicol}
\usepackage{chngcntr}

\DeclareMathOperator*{\Res}{Res}

\renewcommand\thesection{\arabic{section}}
\renewcommand\thesubsection{\thesection.\arabic{subsection}}
\renewcommand\thesubsubsection{\thesubsection.\arabic{subsubsection}}

\renewcommand\thesectiondis{\arabic{section}}
\renewcommand\thesubsectiondis{\thesectiondis.\arabic{subsection}}
\renewcommand\thesubsubsectiondis{\thesubsectiondis.\arabic{subsubsection}}


\hyphenation{op-tical net-works semi-conduc-tor}
\def\inputGnumericTable{}                                 %%

\lstset{
%language=C,
frame=single, 
breaklines=true,
columns=fullflexible
}
\begin{document}

\newcommand{\BEQA}{\begin{eqnarray}}
\newcommand{\EEQA}{\end{eqnarray}}
\newcommand{\define}{\stackrel{\triangle}{=}}
\bibliographystyle{IEEEtran}
\raggedbottom
\setlength{\parindent}{0pt}
\providecommand{\mbf}{\mathbf}
\providecommand{\pr}[1]{\ensuremath{\Pr\left(#1\right)}}
\providecommand{\qfunc}[1]{\ensuremath{Q\left(#1\right)}}
\providecommand{\sbrak}[1]{\ensuremath{{}\left[#1\right]}}
\providecommand{\lsbrak}[1]{\ensuremath{{}\left[#1\right.}}
\providecommand{\rsbrak}[1]{\ensuremath{{}\left.#1\right]}}
\providecommand{\brak}[1]{\ensuremath{\left(#1\right)}}
\providecommand{\lbrak}[1]{\ensuremath{\left(#1\right.}}
\providecommand{\rbrak}[1]{\ensuremath{\left.#1\right)}}
\providecommand{\cbrak}[1]{\ensuremath{\left\{#1\right\}}}
\providecommand{\lcbrak}[1]{\ensuremath{\left\{#1\right.}}
\providecommand{\rcbrak}[1]{\ensuremath{\left.#1\right\}}}
\theoremstyle{remark}
\newtheorem{rem}{Remark}
\newcommand{\sgn}{\mathop{\mathrm{sgn}}}
\providecommand{\abs}[1]{\vert#1\vert}
\providecommand{\res}[1]{\Res\displaylimits_{#1}} 
\providecommand{\norm}[1]{\lVert#1\rVert}
%\providecommand{\norm}[1]{\lVert#1\rVert}
\providecommand{\mtx}[1]{\mathbf{#1}}
\providecommand{\mean}[1]{E[ #1 ]}
\providecommand{\fourier}{\overset{\mathcal{F}}{ \rightleftharpoons}}
%\providecommand{\hilbert}{\overset{\mathcal{H}}{ \rightleftharpoons}}
\providecommand{\system}{\overset{\mathcal{H}}{ \longleftrightarrow}}
	%\newcommand{\solution}[2]{\textbf{Solution:}{#1}}
\newcommand{\solution}{\noindent \textbf{Solution: }}
\newcommand{\cosec}{\,\text{cosec}\,}
\providecommand{\dec}[2]{\ensuremath{\overset{#1}{\underset{#2}{\gtrless}}}}
\newcommand{\myvec}[1]{\ensuremath{\begin{pmatrix}#1\end{pmatrix}}}
\newcommand{\mydet}[1]{\ensuremath{\begin{vmatrix}#1\end{vmatrix}}}
\numberwithin{equation}{subsection}
\makeatletter
\@addtoreset{figure}{problem}
\makeatother
\let\StandardTheFigure\thefigure
\let\vec\mathbf
\renewcommand{\thefigure}{\theproblem}
\def\putbox#1#2#3{\makebox[0in][l]{\makebox[#1][l]{}\raisebox{\baselineskip}[0in][0in]{\raisebox{#2}[0in][0in]{#3}}}}
     \def\rightbox#1{\makebox[0in][r]{#1}}
     \def\centbox#1{\makebox[0in]{#1}}
     \def\topbox#1{\raisebox{-\baselineskip}[0in][0in]{#1}}
     \def\midbox#1{\raisebox{-0.5\baselineskip}[0in][0in]{#1}}
\vspace{3cm}
\title{Assignment 5}
\author{Digjoy Nandi - AI20BTECH11007}
\maketitle
\newpage
\bigskip
\renewcommand{\thefigure}{\theenumi}
\renewcommand{\thetable}{\theenumi}
Download all python codes from 
\begin{lstlisting}

\end{lstlisting}
%
and latex codes from 
%
\begin{lstlisting}

\end{lstlisting}
\section*{\textbf{Problem}}
\textbf{(CSIR UGC NET Maths June 2018-Q103)}-Let X and Y be two random variables with joint probability density function
\begin{align*}
    f(x.y)=
    \begin{cases}
    \cfrac{1}{\pi} & 0\le x^2 + y^2 \le 1\\
    0 & otherwise
    \end{cases}
\end{align*}
Which of the following statements are correct?
\begin{enumerate}
    \item
    X and Y are independent.
    
    \item
    \pr{X>0}=\cfrac{1}{2}
    
    \item
    E(Y)=0
    
    \item
    Cov(X,Y)=0
\end{enumerate}
\section*{\textbf{Solution}}
The marginal PDF of X is given by
\begin{align}
    f_X(x)&=\displaystyle\int\limits_{-\infty}^{\infty} f_{XY}(x,y) dy\\
          &=\displaystyle\int\limits_{-\sqrt{1-x^2}}^{\sqrt{1-x^2}}\cfrac{1}{\pi} dy\\
          &=\cfrac{2\sqrt{1-x^2}}{\pi}
\end{align}
The marginal PDF of Y is given by
\begin{align}
    f_Y(x)&=\displaystyle\int\limits_{-\infty}^{\infty} f_{XY}(x,y) dx\\
          &=\displaystyle\int\limits_{-\sqrt{1-y^2}}^{\sqrt{1-y^2}}\cfrac{1}{\pi} dx\\
          &=\cfrac{2\sqrt{1-y^2}}{\pi}
\end{align}
Now,
\begin{align}
    f_X(x)\times f_Y(x) &=\cfrac{2\sqrt{1-x^2}}{\pi} \times\cfrac{2\sqrt{1-y^2}}{\pi}\\
    &= \cfrac{4(1-x^2)(1-y^2)}{\pi^2}\\
    &\neq \cfrac{1}{\pi}\\
    &\neq f_{XY}(x,y)
\end{align}
Therefore, X and Y are not independent.\\

Now,
\begin{align}
    \pr{X>0}&=\displaystyle\int\limits_{0}^{\infty}f_X(x)dx\\
    &=\displaystyle\int\limits_{0}^{1}\cfrac{2\sqrt{1-x^2}}{\pi} dx\\
    &=\brak{\frac{\arcsin{(x)}+x\sqrt{1-x^2}}{\pi}}_{0}^{1}\\
    &=\cfrac{1}{2}
\end{align}
Therefore, option(2) is correct.\\

Now,
\begin{align}
    E\sbrak{Y}&=\displaystyle\int\limits_{-\infty}^{\infty}yf_Y(y)dy\\
    &=\displaystyle\int\limits_{-1}^{1}\cfrac{2y\sqrt{1-y^2}}{\pi}dy\\
    &=\brak{\frac{-2(1-y^2)^{\frac{3}{2}}}{3\pi}}_{-1}^{1}\\
    &=0
\end{align}
Therefore, option(3) is also correct.\\

Now,
\begin{align}
    E\sbrak{XY}&=\displaystyle\int\limits_{x}\displaystyle\int\limits_{y}xyf_{XY}(x,y)dydx\\
    &=\displaystyle\int\limits_{-1}^{1}\displaystyle\int\limits_{-\sqrt{1-x^2}}^{\sqrt{1-x^2}}\cfrac{xy}{\pi}dydx\\
    &=\cfrac{x}{\pi} \displaystyle\int\limits_{-1}^{1}\brak{\cfrac{y^2}{2}}_{-\sqrt{1-x^2}}^{\sqrt{1-x^2}}dx\\
    &=0
\end{align}
Now,
\begin{align}
    Cov(X,Y)&=E\sbrak{XY}-E\sbrak{X}E\sbrak{Y}\\
    &=0-E\sbrak{X}\times0\\
    &=0
\end{align}
Therefore, option(4) is also correct.\\

\textbf{The correct options are (2),(3) and (4).}

\end{document}

