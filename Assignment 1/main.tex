\documentclass[journal,12pt,twocolumn]{IEEEtran}

\usepackage{setspace}
\usepackage{gensymb}
\singlespacing
\usepackage[cmex10]{amsmath}

\usepackage{amsthm}

\usepackage{mathrsfs}
\usepackage{txfonts}
\usepackage{stfloats}
\usepackage{bm}
\usepackage{cite}
\usepackage{cases}
\usepackage{subfig}

\usepackage{longtable}
\usepackage{multirow}

\usepackage{enumitem}
\usepackage{mathtools}
\usepackage{steinmetz}
\usepackage{tikz}
\usepackage{circuitikz}
\usepackage{verbatim}
\usepackage{tfrupee}
\usepackage[breaklinks=true]{hyperref}
\usepackage{graphicx}
\usepackage{tkz-euclide}

\usetikzlibrary{calc,math}
\usepackage{listings}
    \usepackage{color}                                            %%
    \usepackage{array}                                            %%
    \usepackage{longtable}                                        %%
    \usepackage{calc}                                             %%
    \usepackage{multirow}                                         %%
    \usepackage{hhline}                                           %%
    \usepackage{ifthen}                                           %%
    \usepackage{lscape}     
\usepackage{multicol}
\usepackage{chngcntr}

\DeclareMathOperator*{\Res}{Res}

\renewcommand\thesection{\arabic{section}}
\renewcommand\thesubsection{\thesection.\arabic{subsection}}
\renewcommand\thesubsubsection{\thesubsection.\arabic{subsubsection}}

\renewcommand\thesectiondis{\arabic{section}}
\renewcommand\thesubsectiondis{\thesectiondis.\arabic{subsection}}
\renewcommand\thesubsubsectiondis{\thesubsectiondis.\arabic{subsubsection}}


\hyphenation{op-tical net-works semi-conduc-tor}
\def\inputGnumericTable{}                                 %%

\lstset{
%language=C,
frame=single, 
breaklines=true,
columns=fullflexible
}
\begin{document}

\newcommand{\BEQA}{\begin{eqnarray}}
\newcommand{\EEQA}{\end{eqnarray}}
\newcommand{\define}{\stackrel{\triangle}{=}}
\bibliographystyle{IEEEtran}
\raggedbottom
\setlength{\parindent}{0pt}
\providecommand{\mbf}{\mathbf}
\providecommand{\pr}[1]{\ensuremath{\Pr\left(#1\right)}}
\providecommand{\qfunc}[1]{\ensuremath{Q\left(#1\right)}}
\providecommand{\sbrak}[1]{\ensuremath{{}\left[#1\right]}}
\providecommand{\lsbrak}[1]{\ensuremath{{}\left[#1\right.}}
\providecommand{\rsbrak}[1]{\ensuremath{{}\left.#1\right]}}
\providecommand{\brak}[1]{\ensuremath{\left(#1\right)}}
\providecommand{\lbrak}[1]{\ensuremath{\left(#1\right.}}
\providecommand{\rbrak}[1]{\ensuremath{\left.#1\right)}}
\providecommand{\cbrak}[1]{\ensuremath{\left\{#1\right\}}}
\providecommand{\lcbrak}[1]{\ensuremath{\left\{#1\right.}}
\providecommand{\rcbrak}[1]{\ensuremath{\left.#1\right\}}}
\theoremstyle{remark}
\newtheorem{rem}{Remark}
\newcommand{\sgn}{\mathop{\mathrm{sgn}}}
\providecommand{\abs}[1]{\vert#1\vert}
\providecommand{\res}[1]{\Res\displaylimits_{#1}} 
\providecommand{\norm}[1]{\lVert#1\rVert}
%\providecommand{\norm}[1]{\lVert#1\rVert}
\providecommand{\mtx}[1]{\mathbf{#1}}
\providecommand{\mean}[1]{E[ #1 ]}
\providecommand{\fourier}{\overset{\mathcal{F}}{ \rightleftharpoons}}
%\providecommand{\hilbert}{\overset{\mathcal{H}}{ \rightleftharpoons}}
\providecommand{\system}{\overset{\mathcal{H}}{ \longleftrightarrow}}
	%\newcommand{\solution}[2]{\textbf{Solution:}{#1}}
\newcommand{\solution}{\noindent \textbf{Solution: }}
\newcommand{\cosec}{\,\text{cosec}\,}
\providecommand{\dec}[2]{\ensuremath{\overset{#1}{\underset{#2}{\gtrless}}}}
\newcommand{\myvec}[1]{\ensuremath{\begin{pmatrix}#1\end{pmatrix}}}
\newcommand{\mydet}[1]{\ensuremath{\begin{vmatrix}#1\end{vmatrix}}}
\numberwithin{equation}{subsection}
\makeatletter
\@addtoreset{figure}{problem}
\makeatother
\let\StandardTheFigure\thefigure
\let\vec\mathbf
\renewcommand{\thefigure}{\theproblem}
\def\putbox#1#2#3{\makebox[0in][l]{\makebox[#1][l]{}\raisebox{\baselineskip}[0in][0in]{\raisebox{#2}[0in][0in]{#3}}}}
     \def\rightbox#1{\makebox[0in][r]{#1}}
     \def\centbox#1{\makebox[0in]{#1}}
     \def\topbox#1{\raisebox{-\baselineskip}[0in][0in]{#1}}
     \def\midbox#1{\raisebox{-0.5\baselineskip}[0in][0in]{#1}}
\vspace{3cm}
\title{Assignment 1}
\author{Digjoy Nandi - AI20BTECH11007}
\maketitle
\newpage
\bigskip
\renewcommand{\thefigure}{\theenumi}
\renewcommand{\thetable}{\theenumi}
Download all python codes from 
\begin{lstlisting}
https://github.com/Digjoy12/probability/blob/main/Assignment%201/main.tex
\end{lstlisting}
%
and latex codes from 
%
\begin{lstlisting}
https://github.com/Digjoy12/probability/blob/main/Assignment%201/codes/code.py.py
\end{lstlisting}
\section*{Problem(6.7)}
An electronic assembly consists of two
subsystems, say, A and B. From previous
testing procedures, the following probabilities are assumed to be known:\\
P(A fails) = 0.2\\
P(B fails alone) = 0.15\\
P(A and B fail) = 0.15\\
Evaluate the following probabilities\\
(i) P(A fails alone)\\
(ii) P(A fails—B has failed)\\
\section*{Solution(6.7)}
Given,\\
\begin{align*}
\pr{\text {A fails}}&=\pr{A}&=0.2\\
\pr{\text{B fails alone}}&=\pr{B-A}&=0.15\\
\pr{\text{A and B fails}}&=\pr{AB}&=0.15\\
\end{align*}
 Now,we need to find \pr{\text{A fails alone}}=\pr{A-B}
\begin{enumerate}
\item
\begin{align}
    \tag{6.7.1}
    \pr{A}&=\pr{A-B}+\pr{AB}\\
    \tag{6.7.2}
    \implies \pr{A-B}&=\pr{A}-\pr{AB}\\
    \tag{6.7.3}
    \implies \pr{A-B}&=0.20-0.15\\
    \tag{6.7.4}
    \implies \pr{A-B}&=0.05
\end{align}
Therefore, \pr{\text{A fails alone}}=\pr{A-B}=0.05
\item
Now,finding the probability of B
\begin{align}
    \tag{6.7.5}
    \pr{B-A}&=\pr{B}-\pr{AB}\\
    \tag{6.7.6}
    \implies \pr{B}&=\pr{B-A}+\pr{AB}\\
    \tag{6.7.7}
    \implies \pr{B}&=0.15+0.15\\
    \tag{6.7.8}
    \implies \pr{B}&=0.30
\end{align}
Now, we need to find \pr{\text{A fails$|$B has failed}}=\pr{A|B}
\begin{align}
    \tag{6.7.9}
    \pr{A|B}&=\frac{\pr{AB}}{\pr{B}}\\
    \tag{6.7.10}
    \implies \pr{A|B}&=\frac{0.15}{0.30}\\
    \tag{6.7.11}
    \implies \pr{A|B}&=0.5
\end{align}
Therefore, \pr{\text{A fails$|$ B has failed}}=\pr{A|B}=0.5
\end{enumerate}
\end{document}
