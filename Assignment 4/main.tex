\documentclass[journal,12pt,twocolumn]{IEEEtran}

\usepackage{setspace}
\usepackage{gensymb}
\singlespacing
\usepackage[cmex10]{amsmath}

\usepackage{amsthm}

\usepackage{mathrsfs}
\usepackage{txfonts}
\usepackage{stfloats}
\usepackage{bm}
\usepackage{cite}
\usepackage{cases}
\usepackage{subfig}

\usepackage{longtable}
\usepackage{multirow}

\usepackage{enumitem}
\usepackage{mathtools}
\usepackage{steinmetz}
\usepackage{tikz}
\usepackage{circuitikz}
\usepackage{verbatim}
\usepackage{tfrupee}
\usepackage[breaklinks=true]{hyperref}
\usepackage{graphicx}
\usepackage{tkz-euclide}

\usetikzlibrary{calc,math}
\usepackage{listings}
    \usepackage{color}                                            %%
    \usepackage{array}                                            %%
    \usepackage{longtable}                                        %%
    \usepackage{calc}                                             %%
    \usepackage{multirow}                                         %%
    \usepackage{hhline}                                           %%
    \usepackage{ifthen}                                           %%
    \usepackage{lscape}     
\usepackage{multicol}
\usepackage{chngcntr}

\DeclareMathOperator*{\Res}{Res}

\renewcommand\thesection{\arabic{section}}
\renewcommand\thesubsection{\thesection.\arabic{subsection}}
\renewcommand\thesubsubsection{\thesubsection.\arabic{subsubsection}}

\renewcommand\thesectiondis{\arabic{section}}
\renewcommand\thesubsectiondis{\thesectiondis.\arabic{subsection}}
\renewcommand\thesubsubsectiondis{\thesubsectiondis.\arabic{subsubsection}}


\hyphenation{op-tical net-works semi-conduc-tor}
\def\inputGnumericTable{}                                 %%

\lstset{
%language=C,
frame=single, 
breaklines=true,
columns=fullflexible
}
\begin{document}

\newcommand{\BEQA}{\begin{eqnarray}}
\newcommand{\EEQA}{\end{eqnarray}}
\newcommand{\define}{\stackrel{\triangle}{=}}
\bibliographystyle{IEEEtran}
\raggedbottom
\setlength{\parindent}{0pt}
\providecommand{\mbf}{\mathbf}
\providecommand{\pr}[1]{\ensuremath{\Pr\left(#1\right)}}
\providecommand{\qfunc}[1]{\ensuremath{Q\left(#1\right)}}
\providecommand{\sbrak}[1]{\ensuremath{{}\left[#1\right]}}
\providecommand{\lsbrak}[1]{\ensuremath{{}\left[#1\right.}}
\providecommand{\rsbrak}[1]{\ensuremath{{}\left.#1\right]}}
\providecommand{\brak}[1]{\ensuremath{\left(#1\right)}}
\providecommand{\lbrak}[1]{\ensuremath{\left(#1\right.}}
\providecommand{\rbrak}[1]{\ensuremath{\left.#1\right)}}
\providecommand{\cbrak}[1]{\ensuremath{\left\{#1\right\}}}
\providecommand{\lcbrak}[1]{\ensuremath{\left\{#1\right.}}
\providecommand{\rcbrak}[1]{\ensuremath{\left.#1\right\}}}
\theoremstyle{remark}
\newtheorem{rem}{Remark}
\newcommand{\sgn}{\mathop{\mathrm{sgn}}}
\providecommand{\abs}[1]{\vert#1\vert}
\providecommand{\res}[1]{\Res\displaylimits_{#1}} 
\providecommand{\norm}[1]{\lVert#1\rVert}
%\providecommand{\norm}[1]{\lVert#1\rVert}
\providecommand{\mtx}[1]{\mathbf{#1}}
\providecommand{\mean}[1]{E[ #1 ]}
\providecommand{\fourier}{\overset{\mathcal{F}}{ \rightleftharpoons}}
%\providecommand{\hilbert}{\overset{\mathcal{H}}{ \rightleftharpoons}}
\providecommand{\system}{\overset{\mathcal{H}}{ \longleftrightarrow}}
	%\newcommand{\solution}[2]{\textbf{Solution:}{#1}}
\newcommand{\solution}{\noindent \textbf{Solution: }}
\newcommand{\cosec}{\,\text{cosec}\,}
\providecommand{\dec}[2]{\ensuremath{\overset{#1}{\underset{#2}{\gtrless}}}}
\newcommand{\myvec}[1]{\ensuremath{\begin{pmatrix}#1\end{pmatrix}}}
\newcommand{\mydet}[1]{\ensuremath{\begin{vmatrix}#1\end{vmatrix}}}
\numberwithin{equation}{subsection}
\makeatletter
\@addtoreset{figure}{problem}
\makeatother
\let\StandardTheFigure\thefigure
\let\vec\mathbf
\renewcommand{\thefigure}{\theproblem}
\def\putbox#1#2#3{\makebox[0in][l]{\makebox[#1][l]{}\raisebox{\baselineskip}[0in][0in]{\raisebox{#2}[0in][0in]{#3}}}}
     \def\rightbox#1{\makebox[0in][r]{#1}}
     \def\centbox#1{\makebox[0in]{#1}}
     \def\topbox#1{\raisebox{-\baselineskip}[0in][0in]{#1}}
     \def\midbox#1{\raisebox{-0.5\baselineskip}[0in][0in]{#1}}
\vspace{3cm}
\title{Assignment 3}
\author{Digjoy Nandi - AI20BTECH11007}
\maketitle
\newpage
\bigskip
\renewcommand{\thefigure}{\theenumi}
\renewcommand{\thetable}{\theenumi}
Download all python codes from 
\begin{lstlisting}
https://github.com/Digjoy12/probability/blob/main/Assignment%204/Code/main.py
\end{lstlisting}
%
and latex codes from 
%
\begin{lstlisting}
https://github.com/Digjoy12/probability/blob/main/Assignment%204/main.tex
\end{lstlisting}
\section*{\textbf{Problem}}
\textbf{(GATE CS - 2012 Q.33)} Suppose a fair six-sided die is rolled once. If the value on the die is 1,2, or 3, the die is rolled a second time. What is the probability that the sum total of values that turn up is at least 6?
\begin{enumerate}
    \item 10/21
    \item 5/12
    \item 2/3
    \item 1/6
\end{enumerate}
\section*{\textbf{Solution}}
Let us define a random variable $X \in \cbrak{0,1}$\\
\begin{table}[ht]
    \centering
    \begin{tabular}{|c|c|}
    \hline
    X=0 & Getting 1,2, or 3 on first roll\\
    \hline
    X=1 & Getting the sum total of values at least 6\\
    \hline
    \end{tabular}
    \caption{Random Variables}
    \label{tab:Random Variables}
\end{table}
\newline
Probability of getting 1,2, or 3 on first roll is given by,
\begin{align}
    \pr{X=0}&=\cfrac{3}{6}= \cfrac{1}{2}\\
\end{align}
Probability of getting sum total of 6 on first roll is given by,
\begin{align}   
    \pr{X=1}&= \cfrac{1}{6}\\
\end{align} 
Probability of getting sum total of 6 after getting 1,2,or,3 in first roll is given by,
\begin{align}    
    \pr{X=1|X=0}&=\cfrac{9}{18}= \cfrac{1}{2}
\end{align}

Now,probability of getting sum total of 6 and getting 1,2,or,3 in first roll is given by,
\begin{align}
    \pr{X=1 , X=0}&=\pr{X=1|X=0}\times\pr{X=0}\\
                     &=\cfrac{1}{2} \times \cfrac{1}{2}\\
                     &=\cfrac{1}{4}
\end{align}
\begin{table}[ht]
    \centering
    \begin{tabular}{|c|c|c|c|}
    \hline
    X & X=0 & X=1 & X=1$|$X=0\\
    \hline
    $\pr{X}$ & $\cfrac{1}{2}$ & $\cfrac{1}{6}$ & $\cfrac{1}{2}$\\
    \hline
    \end{tabular}
    \caption{Probability distribution table}
    \label{tab:Probability distribution table}
\end{table}
\newline
The probability that the sum total of values that turn up is at least 6 is given by
\begin{align}
    \pr{X=1 , X=0} + \pr{X=1} &= \cfrac{1}{4} + \cfrac{1}{6}\\
    &=\cfrac{5}{12}
\end{align}

\end{document}
